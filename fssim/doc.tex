\documentclass[a4paper]{article}

%% Language and font encodings
\usepackage[czech]{babel}
%\usepackage[latin]{inputenc}
%\usepackage[utf8]{inputenc}
\usepackage[cp1250]{inputenc}
\usepackage[T1]{fontenc}

%% Sets page size and margins
\usepackage[a4paper,top=3cm,bottom=2cm,left=3cm,right=3cm,marginparwidth=1.75cm]{geometry}

%% Useful packages
\usepackage{amsmath}
\usepackage{graphicx}
\usepackage[colorinlistoftodos]{todonotes}
\usepackage[colorlinks=true, allcolors=blue]{hyperref}

\title{Your Paper}
\author{You}

\begin{document}
\maketitle

\section{Zadání}

Naprogramujte v ANSI C přenositelnou konzolovou aplikaci, která načte ze souboru obraz souborového systému a seznam příkazů, které se mají nad souborovým systémem vykonat.
Program se bude spouštět příkazem: fssim.exe (files) (commands). Symbol (files) zastupuje jméno souboru s obrazem souborového systému a symbol (commands) zastupuje jméno souboru s příkazy (popis vstupních souborů bude uveden dále). Váš program tedy může být během testování spuštěn například takto:
. . . \{ \}>fssim.exe files.txt commands.txt
Výstupem programu bude výstup jednotlivých příkazů vykonaných nad simulovaným souborovým systémem vypsaný do příkazové řádky. Pokud nebudou uvedeny právě dva argumenty, vypište chybové hlášení a stručný návod k použití programu v angličtině podle běžných zvyklostí (viz např. ukázková semestrální práce na webu předmětu Programování v jazyce C). Vstupem programu jsou pouze argumenty na příkazové řádce – interakce s uživatelem pomocí klávesnice
či myši v průběhu práce programu se neočekává.

Hotovou práci odevzdejte v jediném archivu typu ZIP prostřednictvím automatického odevzdávacího a validačního systému. Archiv necht’ obsahuje všechny zdrojové soubory potřebné k přeložení programu, makefile pro Windows i Linux (pro přreklad v Linuxu připravte soubor pojmenovaný makefile a pro Windows makefile.win) a dokumentaci ve formátu PDF vytvořenou v typografickém systému \TeX, resp. \LaTeX. Bude-li některá z částí chybět, kontrolní skript Vaši práci
odmítne.

\section{Analýza úlohy}

\section{Popis impelmentace}

\section{Uživatelská příručka}

\section{Závěr}


\bibliographystyle{alpha}
\bibliography{sample}

\end{document}
